\documentclass[a4paper,12pt]{report}

\usepackage[utf8]{inputenc}
\usepackage[russian]{babel}


\begin{document}

%%%%%%%%%%%%%%%%%%%%%%%%%%%%%%%%%%%%%%%%%%%%%%%%%%

	\begin{titlepage}
		\begin{center}
			СПЕЦИФИКАЦИЯ К КУРСОВОЙ РАБОТЕ

			\vspace{1cm}
    		
			{\huge Программа для автоматической синхронизации файлов в двух каталогах на разных компьютерах.\par}
    		
    		\vspace{1cm}
    		по курсу: ТЕХНОЛОГИИ ПРОГРАММИРОВАНИЯ
		\end{center}
		\vfill
		\begin{center}
		Выполнил: \\
		Логашов Денис\\
		группа №5512
		\end{center}
	\end{titlepage}

%%%%%%%%%%%%%%%%%%%%%%%%%%%%%%%%%%%%%%%%%%%%
\begin{frame}
	
	\section*{Функциональная спецификация}
Программа синхронизации файлов будет иметь две реализации: клиентскую и серверную. Обе реализации используют TCP соединение.\\
Клиентская синхронизация представляет собой синхронизацию файлов между двумя компьютерами. При запуске программы экземпляр получает ip и Port другого компьютера. После подключения компьютеров пользователи выбирают каталог для синхронизации. Затем каждый из экземпляров программ формирует список файлов, находящийся в указанной директории. После чего эти списки передаются между компьютерами. После этого начинаются формироваться списки с недостающими файлами. Принятые списки сравниваются с отправленными. Если указанный в списке файл отсутствует в текущей директории, то в список с недостающими файлами добавляется данный файл. После чего происходит обмен сформированными списками. В ответ на этот список программа начинает пересылать файлы, указанные в списке. Если имеются файлы с одинаковым названием, но с разным содержанием, то в лог пишется сообщение о конфликте.\\
Серверная синхронизация представляет собой связь пользователей с сервером. Каждый пользователь знает адрес сервера. На сервере находится список с уже авторизированными пользователями и соответствующие им директории для синхронизации. При первом подключении пользователь задаёт каталог для синхронизации и загружает все файлы из данного каталога на сервер. При повторном подключении происходит синхронизация пользователя с сервером по принципу клиентской синхронизации, с одним лишь исключением, что файлы с одинаковыми названиями можно будет либо заменить в директории пользователя, либо на сервере(по желанию пользователя).
	\section*{Продвинутая версия}
После того, как пользователь установил директорию, он может изменить её используя специальную команду. Также у пользователей есть возможность дать доступ к своему каталогу другим пользователям. При этом у пользователей появляется дополнительные команды: просмотреть список доступных директорий и синхронизация с указанной директорией.

\end{frame}
%%%%%%%%%%%%%%%%%%%%%%%%%%%%%%%%%%%%%%%%%%%

	

%%%%%%%%%%%%%%%%%%%%%%%%%%%%%%%%%%%%%%%%%%

\end{document}

